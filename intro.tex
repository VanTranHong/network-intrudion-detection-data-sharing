\section{Introduction}
\label{sec:intro}
\begin{intro}

The rapid advancement and expansion of the internet has resulted in 
an explosion in the number of Internet of Things (IoT) devices and a huge increase in the network size. This has resulted in a huge increase in attack surface that attackers can exploit. Cybersecurity Ventures, the worls's leading researcher and page one for the global cyber economy, estimates that the global cybercrime costs will grow by 15\% every year over the next 5 years, reading \$10.5 trillion USD annually by 2025. 

Amongst different cyberattacks, Distributed Denial of Service (DDoS) attacks are one of the biggest concerns. It is the kind of attack which tries to stop legitimate users' access to a specific network resource. The word "distributed" comes from the nature of these attacks, meaning that the attacks are scattered and seem to come from different sources and therefore, is distributed. The fact that the attack is scattered making it extremely challenging to inspect all traffic and differentiate legitimate traffic from anomalous ones.

In order to avoid the costly consequences of DDoS attacks, prevention is obviously better than treatment. However, the effectiveness of prevention depends on how well the Intrusion Detection System(IDS) works. Network-based intrusion detection system (NIDS) is the attack detection mechanism that detects attacks by constantly monitor traffic for malicious and benign behavior. However, detecting DDoS is not easy, especially when novel attacks are generated frequently, making previous attack signatures outdated. One potential solution is sharing data. In this way, entities that have not experienced the new attacks can learn the  signatures of these attacks since others have experienced. This helps them react in time for these attacks and avoid being caught in an unprepared situation.

Data sharing can be of many forms. It can be the sharing of raw data or extracted features or models and so on. Raw data is rarely shared due to its huge privacy and security risks. As such, this study is an initial attempt to explore the feasibility of sharing extracted features to improve Network Intrusion Detection System (NIDS) against DDoS attacks. Our research questions and answers are:
\begin{enumerate}
    \item Can data sharing improve NIDS agaiiinst DDoS?\todo{add results}
    \item What kind of data (benign, malicious or both?) will be helpful to share?\todo{add results}
    \item How much data should be shared?\todo{add results}
    \item Under what circumstances will data sharing be useful?\todo{add results}
    
\end{enumerate}








\end{intro}
